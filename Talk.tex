\documentclass[hyperref={pdfpagelabels=false}]{beamer}
\usepackage{CJKutf8}
\usepackage[english]{babel}
\usepackage{xcolor}
\usepackage{lmodern}
\usepackage{amssymb}
\usepackage[makeroom]{cancel} %for crossing symbols
\usepackage{leftindex} %For leftindex, making it possible to have nicely aligned left subscripts
%\usepackage{calligra}
%\DeclareMathAlphabet{\mathcalligra}{T1}{calligra}{m}{n} %For small \mathcal letters
\makeatletter
\DeclareFontEncoding{LS1}{}{}
\DeclareFontSubstitution{LS1}{stix}{m}{n}
\DeclareMathAlphabet{\mathKel}{LS1}{stixscr}{m}{n}
\DeclareMathAlphabet{\mathcal}{LS1}{stixscr}{m}{n}
\usepackage{amsthm}
\usepackage{amsmath}
%\usepackage{mathabx}
\usepackage{stmaryrd}
\usepackage{amsbsy}
\usepackage{dsfont}
\usepackage{mathtools} %für mathclap und coloneqq
%\usepackage{amsbsy}
\usepackage{mleftright} %Distanz zu \left \right weg
\usepackage{tikz-cd}

\usepackage{tabularx} %Automatic line break of tables using X instead c l r
%\usepackage{longtable} %table auf mehreren Seiten
%\usepackage{ltxtable} %Combination of both above
\usepackage{xcolor, colortbl}

%Für die ganzen Diagramme
\usepackage{pgfplots}
\usepackage{graphicx} %Für raisebox, vertical displacement of figures
\usetikzlibrary{decorations.markings, decorations.text,calc,arrows.meta}

\definecolor{Gray}{gray}{0.85}
%\usepackage[style=authortitle-icomp]{biblatex}
%\usepackage[babel,german=guillemets]{csquotes}

\setcounter{tocdepth}{1}
%\setcounter{tocdepth}{5}
%\setcounter{secnumdepth}{4}
%\setcounter{secnumdepth}{5}
\usepackage[backend=biber, style=numeric]{biblatex}
\addbibresource{Literatur.bib}
\newcommand{\footlineextra}[1]{
    \begin{tikzpicture}[remember picture,overlay]
        \node[yshift=2ex,anchor=south west] at (current page.south west) {\usebeamerfont{author in head/foot}\hspace{2ex}#1};
    \end{tikzpicture}
}

\newcommand\insertreferences{}
\setbeamertemplate{footline}{%
  \leavevmode%
  \hbox{%
  \begin{beamercolorbox}[wd=.09\paperwidth, ht=5ex,dp=1ex,center, sep=1.4ex]{author in head/foot}%
    \usebeamerfont{author in head/foot}
		%\vfill
		%Sources
		%\vfill
		Sources
  \end{beamercolorbox}%
  \begin{beamercolorbox}[wd=.91\paperwidth,ht=5ex,dp=1ex,center]{title in head/foot}%
    \usebeamerfont{title in head/foot}
    \insertreferences

  \end{beamercolorbox}
}
}


\title{Classification of neighbourhoods of leaves of singular foliations}   
\subtitle{joint work with Camille Laurent-Gengoux}   
\author{Simon-Raphael Fischer} 
\institute{\begin{CJK*}{UTF8}{bkai}國家理論科學研究中心\end{CJK*}}
\date{4 September 2023} 
%\date{Le lundi 31 mai 2021} 

% zusaetzlich ist das usepackage{beamerthemeshadow} eingebunden 
%\usepackage{beamerthemeIlmenau}
%\usepackage{beamerthemeshadow}
\usepackage{beamerthemeDarmstadt}

%  \beamersetuncovermixins{\opaqueness<1>{25}}{\opaqueness<2->{15}}
%  sorgt dafuer das die Elemente die erst noch (zukuenftig) kommen 
%  nur schwach angedeutet erscheinen 
\beamersetuncovermixins{\opaqueness<1>{25}}{\opaqueness<2->{15}}
% klappt auch bei Tabellen, wenn teTeX verwendent wird\ldots

\beamertemplatenavigationsymbolsempty %Damit sind die kleinen Navigationssymbole unten weg

%\usesectionheadtemplate{}{}
%\usesubsectionheadtemplate{}{}

\def\be{\begin{equation}}
\def\ee{\end{equation}}
\def\bs{\begin{subequations}}
\def\es{\end{subequations}}
\def\ba#1\ea{\begin{align}#1\end{align}}
\def\bes{\begin{equation*}}
\def\ees{\end{equation*}}
\def\bas#1\eas{\begin{align*}#1\end{align*}}

\AtBeginEnvironment{remark}{%
  \setbeamercolor{block title}{use=example text,fg=black,bg=yellow!75!black}
  \setbeamercolor{block body}{parent=normal text,use=block title example,bg=yellow!10}
}

\renewcommand{\qedsymbol}{}
\theoremstyle{plain}
\newtheorem{conjecture}[theorem]{Conjecture}
\newtheorem{proposition}[theorem]{Proposition}
%\newtheorem{definition}[theorem]{Definition}
\theoremstyle{remark}
\newtheorem*{remark}{Remarks}
\newtheorem*{idea}{Idea}
\newtheorem*{motivation}{Motivation}
\newtheorem*{summary}{Summary}
\newtheorem*{situation}{Situation}
\newtheorem*{lab}{Situation: Lie algebra bundles}
\newtheorem*{question}{Question}
\newtheorem*{fieldredefinition}{Field Redefinition}
\newtheorem*{construction}{Construction}
\newtheorem*{aim}{Aim}
\newtheorem*{BackToTheRoots}{Sketch of construction}

\AtBeginEnvironment{BackToTheRoots}{%
  \setbeamercolor{block title}{use=example text,fg=black,bg=pink!75!black}
  \setbeamercolor{block body}{parent=normal text,use=block title example,bg=pink!10}
}

%\theoremstyle{definition}
%\newtheorem{definition}[theorem]{Definition}
%\newtheorem*{SecondIn}{Second Inequality}

%mathrm mit mathup ersetzen, damit die font passt
\renewcommand\familydefault{\sfdefault} %comment to see the difference
\DeclareMathAlphabet      {\mathup}{OT1}{\familydefault}{m}{n}


\begin{document}


\begin{frame}
\thispagestyle{empty}
\titlepage
\end{frame} 


{
\setbeamertemplate{footline}{}
\begin{frame}
\frametitle{Table of contents}
\tableofcontents
\end{frame} 
}

\section{Singular Foliations}

{
\setbeamertemplate{footline}
{
  \leavevmode%
  \hbox{%
  \begin{beamercolorbox}[wd=.09\paperwidth,ht=2.25ex,dp=1ex,center]{author in head/foot}%
    \usebeamerfont{author in head/foot}
		Sources
  \end{beamercolorbox}%
  \begin{beamercolorbox}[wd=.91\paperwidth,ht=2.25ex,dp=1ex,center]{title in head/foot}%
    \usebeamerfont{title in head/foot}
		Right diagram made by Mark J.D.\ Hamilton.
  \end{beamercolorbox}%
  %\begin{beamercolorbox}[wd=.175\paperwidth,ht=2.25ex,dp=1ex,right]{date in head/foot}%
    %\usebeamerfont{date in head/foot}\insertshortdate{}\hspace*{2em}
    %%\insertframenumber{} 
		%%/ \inserttotalframenumber\hspace*{2ex} 
  %\end{beamercolorbox}
	}%
  \vskip0pt%
}
\begin{frame}
\begin{tikzpicture}
\coordinate (O) at (0,0);
%\foreach \j in {1,...,3} \draw (O) circle (3.5-\j);
%\foreach \k/\text in {0/Should be here any!,1/There is a way?,2/Wee} \draw[decoration={text along path,reverse path,text align={align=center},text={\text}},decorate] (2.6-\k,0) arc (0:180:2.6-\k);
\foreach \k in {1,...,4}\pgfmathparse{12*\k} \draw[fill=blue!\pgfmathresult] (O) circle (3.6-0.8*\k) node at (0, 3) {$\mathbb{S}^n$};
%\foreach \k/\text in {0/{$S^n$},1/,2/,3/} \draw[decoration={text along path,reverse path,text align={align=center},text={\text}},decorate] (2.9-0.8*\k,0) arc (0:180:2.9-0.8*\k);
\fill (O) circle[radius=2pt];
\begin{scope}[xshift=2.2cm, yshift=-1.8cm]
\begin{axis}[ scale = 1,
            hide axis,
            %axis lines=middle,
%            axis on top,
%            axis line style={blue,dashed,thick},
%            ymin=-2,ymax=2,
%            xmin=-2,xmax=2,
%            zmin=-2,zmax=2,
            samples=40,
            domain=0:360,
            y domain=0:1.25,clip=false
        ]
        \addplot3 [surf, shader=flat, draw=black, fill=gray!10!white, z buffer=sort]
           ({sin(x)*y}, {cos(x)*y}, {(y^2-1)^2});
        \draw[blue,thick,dashed] (axis cs:0,0,0) -- (axis cs:1,0,0)
                    node[below,font=\footnotesize]{};
        \draw[blue,thick,-stealth] (axis cs:1,0,0) -- (axis cs:1.3,0,0)
                    node[above,font=\footnotesize]{};
        \draw[blue,thick,dashed] (axis cs:0,0,0) -- (axis cs:0,-1,0)
                    node[left=2mm,font=\footnotesize]{}; %{Label} am Ende 
        \draw[blue,thick,-stealth] (axis cs:0,-1,0) -- (axis cs:0,-1.5,0)
                    node[right=1mm,font=\footnotesize]{};
        \draw[blue,thick,dashed] (axis cs:0,0,0) -- (axis cs:0,0,1)
                    %node[left=2mm,font=\footnotesize]{$\phi_{\text{RE}}$}
                    ;
        \draw[blue,thick,-stealth] (axis cs:0,0,1) -- (axis cs:0,0,1.3);
\end{axis}
\end{scope}
%\draw[line width=2mm,>={Triangle[length=3mm,width=5mm]},->] (2.6,0) -- (3.8,0);
\end{tikzpicture}
\end{frame}
}

{
\setbeamertemplate{footline}{}
\begin{frame}
\textbf{Singular Foliations:}

\begin{itemize}
	\item Gauge Theory \newline 
	(Ex.: Singular foliation $\leftrightarrow$ Symmetry breaking $\rightarrow$ Higgs mechanism)
	\item Poisson Geometry \newline (Singular foliation of symplectic leaves)
	\item Lie groupoids and algebroids
	\item Dirac structures
	\item Generalised complex manifolds
	\item Non-commutative geometry
	\item $\dotsc$
\end{itemize}

\end{frame}
}

\renewcommand\insertreferences{{\tiny  Peter Stefan, Accessible sets, orbits, and foliations with singularities. \textit{Proc.\ London Math.\ Soc.}, 29, 1974.
\newline
Héctor J. Sussmann, Orbits of families of vector fields and integrability of distributions. \textit{Trans.\ Amer.\ Math.\ Soc.}, 180, 1973}}

\subsection{Definition}

\begin{frame}
\begin{definition}[Smooth singular foliation]
A \textbf{smooth singular foliation $\mathcal{F}$} on a smooth manifold is a subspace of $\mathfrak{X}_c(M)$ so that
\begin{itemize}
	\item it is \textbf{involutive},
	\item it is \textbf{stable under $C^\infty(M)$-multiplication},
	\item it is \textbf{locally finitely generated}.
\end{itemize}
\end{definition}
\end{frame}

\begin{frame}
\begin{definition}[Smooth singular foliation]
A \textbf{smooth singular foliation $\mathcal{F}$} on a smooth manifold is a subspace of $\mathfrak{X}_c(M)$ so that
\begin{itemize}
	\item it is \textbf{involutive}, \textit{i.e.\ $[\mathcal{F}, \mathcal{F}] \subset \mathcal{F}$},
	\item it is \textbf{stable under $C^\infty(M)$-multiplication},
	\item it is \textbf{locally finitely generated}.
\end{itemize}
\end{definition}
\end{frame}

\begin{frame}
\begin{definition}[Smooth singular foliation]
A \textbf{smooth singular foliation $\mathcal{F}$} on a smooth manifold is a subspace of $\mathfrak{X}_c(M)$ so that
\begin{itemize}
	\item it is \textbf{involutive}, \textit{i.e.\ $[\mathcal{F}, \mathcal{F}] \subset \mathcal{F}$},
	\item it is \textbf{stable under $C^\infty(M)$-multiplication}, \textit{i.e.\ $fX \in \mathcal{F}$ for all $f \in C^\infty(M)$ and $X \in \mathcal{F}$},
	\item it is \textbf{locally finitely generated}.
\end{itemize}
\end{definition}
\end{frame}

\begin{frame}
\begin{definition}[Smooth singular foliation]
A \textbf{smooth singular foliation $\mathcal{F}$} on a smooth manifold is a subspace of $\mathfrak{X}_c(M)$ so that
\begin{itemize}
	\item it is \textbf{involutive}, \textit{i.e.\ $[\mathcal{F}, \mathcal{F}] \subset \mathcal{F}$},
	\item it is \textbf{stable under $C^\infty(M)$-multiplication}, \textit{i.e.\ $fX \in \mathcal{F}$ for all $f \in C^\infty(M)$ and $X \in \mathcal{F}$},
	\item it is \textbf{locally finitely generated}, \textit{i.e.\ around each $p \in M$ there is an open neighbourhood $U$ and a finite family $\mleft( X^i \mright)_i^{r}$ ($X^i \in \mathcal{F}$) such that for all $X \in \mathcal{F}$ there are $f_i \in C^\infty(M)$ satisfying on $U$}.
	\bas
	X = \sum_i f_i X^i.
	\eas
\end{itemize}
\end{definition}
\end{frame}

\begin{frame}
\begin{remark}[Leaves]
Following the flows in $\mathcal{F}$, this gives rise to a partition of connected immersed submanifolds in $M$.
\end{remark}

\begin{figure}[htbp]
	\centering
		\includegraphics[width=.70\textwidth]{Foliation example.png}
	\label{fig:Foliation example}
\end{figure}


\end{frame}

%{
%\setbeamertemplate{footline}
%{
  %\leavevmode%
  %\hbox{%
  %\begin{beamercolorbox}[wd=.09\paperwidth,ht=2.25ex,dp=1ex,center]{author in head/foot}%
    %\usebeamerfont{author in head/foot}
		%Sources
  %\end{beamercolorbox}%
  %\begin{beamercolorbox}[wd=.91\paperwidth,ht=2.25ex,dp=1ex,center]{title in head/foot}%
    %\usebeamerfont{title in head/foot}
		%Private communication with Camille Laurent-Gengoux.
  %\end{beamercolorbox}%
  %%\begin{beamercolorbox}[wd=.175\paperwidth,ht=2.25ex,dp=1ex,right]{date in head/foot}%
    %%\usebeamerfont{date in head/foot}\insertshortdate{}\hspace*{2em}
    %%%\insertframenumber{} 
		%%%/ \inserttotalframenumber\hspace*{2ex} 
  %%\end{beamercolorbox}
	%}%
  %\vskip0pt%
%}
%%\begin{frame}{Why finitely generated?}
%
%\begin{figure}[htbp]
	%\centering
		%\includegraphics[width=0.50\textwidth]{Infinite comb.png}
	%\caption{Infinite Comb}
	%\label{fig:Infinite comb}
%\end{figure}
%\end{frame}
%}

\subsection{Idea: Relation to gauge theory}

\renewcommand\insertreferences{{\tiny Camille Laurent-Gengoux and Leonid Ryvkin, The neighborhood of a singular leaf. \newline \textit{Journal de l’{\'E}cole polytechnique-Math{\'e}matiques}, 8, 2021.}}

\begin{frame}
\begin{figure}[htbp]
	\centering
		\includegraphics[width=1.00\textwidth]{Foliation connection.png}
	\caption{$\mathcal{F}$-connections}
	\label{fig:Foliation connection}
\end{figure}

\end{frame}

\begin{frame}
\begin{figure}[htbp]
	\centering
		\includegraphics[width=0.50\textwidth]{Foliation connection.png}
	\caption{$\mathcal{F}$-connections}
	\label{fig:Foliation connection Zwei}
\end{figure}

\begin{theorem}[$\mathcal{F}$-connections]
There is a connection on the normal bundle of a leaf $L$:
\begin{itemize}
	\item Horizontal vector fields are in $\mathcal{F}$.
	\item Parallel transport $\mathup{PT}_\gamma$ has values in $\mathup{Sym}(\tau_l, \tau_{l^\prime})$.
	\item For a contractible loop $\gamma_0$ at $l$: $\mathup{PT}_{\gamma_0}$ values in $\mathup{Inn}(\tau_l)$.
\end{itemize}
\end{theorem}

\end{frame}

\section{Curved Yang-Mills gauge theory}
\subsection{Principal bundles based on Lie group bundle actions}
{
\setbeamertemplate{footline}{}
\begin{frame}
\textbf{Curved Yang-Mills gauge theories:}
\begin{table}[h!]
	\centering
		\begin{tabular}{c c} 
			%\rowcolor{gray}
			Classical & Curved \\
			%Infinitesimal & Lie algebra $\mathfrak{g}$ & LAB\footnote{LAB = Lie algebra bundle} $\mathcal{g}$ \\
			%\rowcolor{Gray}
			Lie group $G$ & \textcolor[rgb]{1,0.41,0.13}{Lie group bundle
			%\footnote{LGB = Lie group bundle} 
			$\mathcal{G}$}
		\end{tabular}
\end{table}

\begin{center}
	\begin{tikzcd}[ampersand replacement=\&]
	G \arrow{r} \& \mathcal{G} \arrow{d} \\
	\& L
	\end{tikzcd}
\end{center}

\end{frame}
}

\renewcommand\insertreferences{{\tiny  K. Mackenzie. General Theory of Lie Groupoids and Algebroids. \newline \textit{London Mathematical Society Lecture Note Series}, 213, 2005.}}

\begin{frame}
\begin{definition}[LGB actions]
\begin{center}
	\begin{tikzcd}[ampersand replacement=\&, column sep = small, row sep = small]
	 \& \mathcal{G} \arrow[bend right]{dl} \arrow{d}{\pi_{\mathcal{G}}} \\
	\mathcal{T} \arrow{r}{\phi} \& L
	\end{tikzcd}
\end{center}
A \textbf{right-action of $\mathcal{G}$ on $\mathcal{T}$} is a smooth map 
%\bas
$\mathcal{T} * \mathcal{G} \coloneqq \mathcal{T} \leftindex^{}_{\phi}\times_{\pi_{\mathcal{G}}} \mathcal{G} \to \mathcal{T}$,
$(t, g) \mapsto t \cdot g$,
%\eas
satisfying the following properties:
\ba\label{InvarianceOffUnderGAction}
\phi(t \cdot g) &= \phi(t),\\
(t \cdot g) \cdot h &= t \cdot (gh),\\
t \cdot e_{\phi(t)} &= p
\ea
for all $t \in \mathcal{T}$ and $g, h \in \mathcal{G}_{\phi(t)}$, where $e_{\phi(t)}$ is the neutral element of $\mathcal{G}_{\phi(t)}$.
\end{definition}
\end{frame}

{
\setbeamertemplate{footline}{}
\begin{frame}
\begin{definition}[Ehresmann/Yang-Mills connection, {[C.\ L.-G., S.-R.\ F.]}]
A surjective submersion $\pi_{\mathcal{T}} \colon \mathcal{T} \to L'$ so that one has a commuting diagram
\begin{center}
	\begin{tikzcd}[ampersand replacement=\&]
	\mathcal{T} \arrow[swap]{d}{\pi_{\mathcal{T}}} \arrow{dr}{\phi} \& \arrow[bend right]{l} \mathcal{G} \arrow{d}{\pi_{\mathcal{G}}}
    %\arrow{l} G 
    \\
	L' \arrow{r}{\Psi} \& L
	\end{tikzcd}
\end{center}
\begin{enumerate}
	\item \textbf{Ehresmann connection:} $\mathcal{G}$ preserving $\pi_{\mathcal{T}}$ and
	\bas
	\mathup{PT}_\gamma^{\mathcal{T}}(t \cdot g)
	&=
	\mathup{PT}_\gamma^{\mathcal{T}}(t)
	\cdot
	\mathup{PT}_{\Psi\circ\gamma}^{\mathcal{G}}(g)
	\eas
	\item \textbf{Yang-Mills connection:} Additionally
	\bas
	\mathup{PT}_{\gamma_0}^{\mathcal{T}}(t)
	&=
	t \cdot g_{\gamma_0}
	\eas
	for some $g_{\gamma_0} \in \mathcal{G}_{\phi(t)}$, where $\gamma_0$ is a contractible loop.
\end{enumerate}
\end{definition}
\end{frame}

\begin{frame}
\begin{definition}[Multiplicative YM connection, {[S.-R.\ F.]}]
On $\mathcal{G}$ there is also the notion of \textbf{multiplicative Yang-Mills connections}, that is,
\bas
\mathup{PT}_\gamma^{\mathcal{G}}(q \cdot g)
&=
\mathup{PT}_\gamma^{\mathcal{G}}(q)
\cdot
\mathup{PT}_{\gamma}^{\mathcal{G}}(g),
\\
\mathup{PT}_{\gamma_0}^{\mathcal{T}}(t)
&=
g_{\gamma_0} \cdot t \cdot g_{\gamma_0}^{-1}
\eas
\end{definition}
\end{frame}
}

\renewcommand\insertreferences{{\tiny Ieke Moerdijk, Janez Mrcun. Introduction to Foliations and Lie Groupoids. \newline \textit{Cambridge Studies in Advanced Mathematics 91, Cambridge University Press, Cambridge}, 2003}}

\begin{frame}
\begin{definition}[Principal bundle]
\begin{center}
	\begin{tikzcd}[ampersand replacement=\&]
	\mathcal{P} \arrow{d}{\pi_{\mathcal{P}}} \arrow{dr}{\phi} \& \arrow[bend right]{l} \mathcal{G} \arrow{d}{\pi_{\mathcal{G}}}
    %\arrow{l} G 
    \\
	L' \arrow{r}{\Psi} \& L
	\end{tikzcd}
\end{center}
A surjective submersion $\pi_{\mathcal{P}}\colon \mathcal{P} \to L'$, with $\mathcal{G}$-action
\bas
\begin{matrix}
	\textcolor[rgb]{1,0,0}{\xcancel{\mathcal{P} \times G}} &\to \mathcal{P} \\
	\mathcal{P} * \mathcal{G} &
\end{matrix}
\eas
simply transitive on $\pi_{\mathcal{P}}$-fibres of $\mathcal{P}$, and "suitable" atlas.
\end{definition}
\end{frame}

%\subsection{Connection as horizontal distribution}

%{
%\setbeamertemplate{footline}{}
%\begin{frame}{Connection on $\mathcal{P}$: Idea}
%\begin{figure}
%%\caption{Pushforwards via right-multiplication of tangent vectors}  
%%\label{figure:fibre bundle}
%\centering
%\begin{tikzpicture}[scale=0.55]
%\draw (0,-3.5) to[out=10,in=170] (4.25,-3.5) to[out=350,in=190] (8.5,-3.5);
%\draw (1.25,2.5) to[out=280,in=90](1.5,0) to[out=270,in=80] (1.25,-2.5);
%\filldraw [fill=gray!20!white, draw=white] (3.5,2.5) to[out=280,in=90](3.75,0) to[out=270,in=80] (3.5,-2.5) -- (5,-2.5)  to[out=80,in=270](5.25,0) to[out=90,in=280] (5,2.5) -- cycle;
%\draw (4.25,2.5) to[out=280,in=95](4.41,1) to[out=275,in=85] (4.41,-1) to[out=265,in=80] (4.25,-2.5); %draw with middle section for precise point placement
%
%\draw (4,-1) node {\textcolor[rgb]{1,0,0}{$p$}};
%\draw [<-, draw=blue] (5,1) to[out=275,in=85] (5,-1);
%\draw (5.5,0) node {\textcolor[rgb]{0,0,1}{$\cdot g$}};
%\filldraw [fill=red, draw=red] (4.41,1) circle (1pt);
%\filldraw [fill=red, draw=red] (4.41,-1) circle (1pt);
%\draw (5.75,2.5) to[out=280,in=90](6,0) to[out=270,in=80] (5.75,-2.5);
%\draw (7.25,2.5) to[out=280,in=90](7.5,0) to[out=270,in=80] (7.25,-2.5);
%\draw (2.75,2.475) to[out=280,in=90](3,0) to[out=270,in=80] (2.75,-2.475);
%
%%\draw (4.75,2) node {$\mathcal{P}_x$};
%%\draw (4.95,1.5) node {$\cong G$};
%\draw (-0.5,-3.5) node {$L^\prime$};
%\draw (19.5,-3.5) node {$L$};
%%\draw (5,2) node {$\mathcal{P}_U$};
%\draw (3.5,-3.4) node {$($};
%\filldraw [fill=red, draw=red] (4.25,-3.5) circle (1pt);
%\draw (4.25,-4.5) node {\textcolor[rgb]{1,0,0}{$x \coloneqq \pi_{\mathcal{P}}(p)$}};
%\draw (5,-3.6) node {$)$};
%%\draw (4.25, -3) node {$U$};
%\draw (0,0) node {$\mathcal{P}$};
%\draw [->] (0,-0.4) -- (0,-3.1);
%\draw (0.7,-1.75) node {$\pi_{\mathcal{P}}$};
%\draw (3.5,1) node {\textcolor[rgb]{1,0,0}{$p \cdot g$}};
%%%%%%%%%%%%%%%%%%%%%%%%%%%%%%%%%%%%%%%%%%%%%%%%%%%%%%%%%%%%%%%%%%%%%%%%%%%%%%%%%%%%%
%\draw (10.5,-3.5) to[out=10,in=170] (14.75,-3.5) to[out=350,in=190] (19,-3.5); %hori M
%\filldraw [fill=gray!20!white, draw=white] (14,2.5) to[out=280,in=90](14.25,0) to[out=270,in=80] (14,-2.5) -- (15.5,-2.5)  to[out=80,in=270](15.75,0) to[out=90,in=280] (15.5,2.5) -- cycle; %gray area
%\draw (11.75,2.5) to[out=280,in=90](12,0) to[out=270,in=80] (11.75,-2.5); %vert line 1
%\draw (14.75,2.5) to[out=280,in=90](15,0) to[out=270,in=80] (14.75,-2.5); %3
%\draw (16.25,2.5) to[out=280,in=90](16.5,0) to[out=270,in=80] (16.25,-2.5); %4
%\draw (17.75,2.5) to[out=280,in=90](18,0) to[out=270,in=80] (17.75,-2.5); %5
%
%\draw (13.25,2.475) to[out=280,in=90](13.5,0) to[out=270,in=80] (13.25,-2.475); %2
%\filldraw [fill=blue, draw=blue] (15,0) circle (1pt);
%\draw (15.5,0) node {\textcolor[rgb]{0,0,1}{$g$}};
%
%\draw (19,0) node {$\mathcal{G}$}; %These three lines are for LGB projection
%\draw [->] (19,-0.4) -- (19,-3.1);
%%\draw (18.5,-1.75) node {$\pi_{\mathcal{G}}$};
%
%%\draw (15.5,2) node {$\mathcal{G}_U$};
%\draw (14,-3.4) node {$($};
%\filldraw [fill=red, draw=red] (14.75,-3.5) circle (1pt);
%\draw (14.75,-4.5) node {\textcolor[rgb]{1,0,0}{$y \coloneqq \phi(p)$}};
%\draw (15.5,-3.6) node {$)$};
%%\draw (14.75, -3) node {$U$};
%
%\path[<-] (8.5,0) edge [bend left] (11,0);
%\end{tikzpicture}
%\end{figure}
%\pause
%But:
%\bas
%&&&r_g: \mathcal{P}_x \to \mathcal{P}_x\\
%&\Rightarrow&&\mathrm{D}_pr_g \text{ only defined on vertical structure}
%\eas
%\end{frame}
%%%%%%%%%%%%%%%%%%%%%%%%%%%%%%%%%%%%%%%%%%%%%%%%%%%%%%%%%%%%%%%%%%%%%%%%%%%%%%%%%%%%%%%%%%%%%%%%%%%%%%%%%%%%%%%%%%%%%%%%%%%%%%%%%%%%%%%%%%%%%%%%%%%%%%%%%%%%%%%%%%%%%%%%%%%%%%%%%%%%%%%%%%%%%%%%%%%%%%%%%%%%%%%%%%%%%%%%%%%%%%%%%%%%%%%%%%%%%%%%%%%%%%%%%%%%%%%%%%%%%%%%%%%%%%%%%%
%\begin{frame}{Connection on $\mathcal{P}$: Idea}
%\begin{figure}
%%\caption{Pushforwards via right-multiplication of tangent vectors}  
%%\label{figure:fibre bundle part two}
%\centering
%\begin{tikzpicture}[scale=0.53]
%\draw (0,-3.5) to[out=10,in=170] (4.25,-3.5) to[out=350,in=190] (8.5,-3.5);
%\draw (1.25,2.5) to[out=280,in=90](1.5,0) to[out=270,in=80] (1.25,-2.5);
%\filldraw [fill=gray!20!white, draw=white] (3.5,2.5) to[out=280,in=90](3.75,0) to[out=270,in=80] (3.5,-2.5) -- (5,-2.5)  to[out=80,in=270](5.25,0) to[out=90,in=280] (5,2.5) -- cycle;
%\draw (4.25,2.5) to[out=280,in=95](4.41,1) to[out=275,in=85] (4.41,-1) to[out=265,in=80] (4.25,-2.5); %draw with middle section for precise point placement
%
%\draw (4,-1) node {\textcolor[rgb]{1,0,0}{$p$}};
%\draw [<-, draw=blue] (5,1) to[out=275,in=85] (5,-1);
%\draw [<-, draw=blue] (4.8,1) to[out=275,in=85] (4.8,-1);
%\draw [<-, draw=blue] (4.6,1) to[out=275,in=85] (4.6,-1);
%\draw (5.5,0) node {\textcolor[rgb]{0,0,1}{$\cdot \sigma$}};
%\filldraw [fill=red, draw=red] (4.41,1) circle (1pt);
%\filldraw [fill=red, draw=red] (4.41,-1) circle (1pt);
%\draw (5.75,2.5) to[out=280,in=90](6,0) to[out=270,in=80] (5.75,-2.5);
%\draw (7.25,2.5) to[out=280,in=90](7.5,0) to[out=270,in=80] (7.25,-2.5);
%\draw (2.75,2.475) to[out=280,in=90](3,0) to[out=270,in=80] (2.75,-2.475);
%
%%\draw (4.75,2) node {$\mathcal{P}_x$};
%%\draw (4.95,1.5) node {$\cong G$};
%\draw (-0.5,-3.5) node {$L^\prime$};
%\draw (19.5,-3.5) node {$L$};
%%\draw (5,2) node {$\mathcal{P}_U$};
%\draw (3.5,-3.4) node {$($};
%\filldraw [fill=red, draw=red] (4.25,-3.5) circle (1pt);
%\draw (4.25,-4.5) node {\textcolor[rgb]{1,0,0}{$x \coloneqq \pi_{\mathcal{P}}(p)$}};
%\draw (5,-3.6) node {$)$};
%%\draw (4.25, -3) node {$U$};
%\draw (0,0) node {$\mathcal{P}$};
%\draw [->] (0,-0.4) -- (0,-3.1);
%\draw (0.7,-1.75) node {$\pi_{\mathcal{P}}$};
%\draw (3.5,1) node {\textcolor[rgb]{1,0,0}{$p \cdot \sigma_y$}};
%%%%%%%%%%%%%%%%%%%%%%%%%%%%%%%%%%%%%%%%%%%%%%%%%%%%%%%%%%%%%%%%%%%%%%%%%%%%%%%%%
%\draw (10.5,-3.5) to[out=10,in=170] (14.75,-3.5) to[out=350,in=190] (19,-3.5); %hori M
%\filldraw [fill=gray!20!white, draw=white] (14,2.5) to[out=280,in=90](14.25,0) to[out=270,in=80] (14,-2.5) -- (15.5,-2.5)  to[out=80,in=270](15.75,0) to[out=90,in=280] (15.5,2.5) -- cycle; %gray area
%\draw (11.75,2.5) to[out=280,in=90](12,0) to[out=270,in=80] (11.75,-2.5); %vert line 1
%\draw (14.75,2.5) to[out=280,in=90](15,0) to[out=270,in=80] (14.75,-2.5); %3
%\draw (16.25,2.5) to[out=280,in=90](16.5,0) to[out=270,in=80] (16.25,-2.5); %4
%\draw (17.75,2.5) to[out=280,in=90](18,0) to[out=270,in=80] (17.75,-2.5); %5
%
%\draw (13.25,2.475) to[out=280,in=90](13.5,0) to[out=270,in=80] (13.25,-2.475); %2
%\draw [draw=blue] (14.25,0) to[out=10,in=170] (15,0) to[out=350,in=190] (15.75,0);
%%\filldraw [fill=blue, draw=blue] (15,0) circle (1pt);
%\draw (13.9,0) node {\textcolor[rgb]{0,0,1}{$\sigma$}};
%
%\draw (19,0) node {$\mathcal{G}$}; %These three lines are for LGB projection
%\draw [->] (19,-0.4) -- (19,-3.1);
%%\draw (18.5,-1.75) node {$\pi_{\mathcal{G}}$};
%
%%\draw (15.5,2) node {$\mathcal{G}_U$};
%\draw (14,-3.4) node {$($};
%\filldraw [fill=red, draw=red] (14.75,-3.5) circle (1pt);
%\draw (14.75,-4.5) node {\textcolor[rgb]{1,0,0}{$y \coloneqq \phi(p)$}};
%\draw (15.5,-3.6) node {$)$};
%%\draw (14.75, -3) node {$U$};
%
%\path[<-] (8.5,0) edge [bend left] (11,0);
%\end{tikzpicture}
%\end{figure}
%
%\bas
%\text{Use } \sigma \in \Gamma(\mathcal{G}): r_\sigma(p) \coloneqq p \cdot \sigma_{y}
%\eas
%
%\pause
%
%\begin{remark}[Problem!]
%Ambiguity in the choice of $\sigma$
%$\Rightarrow$ Fix a horizontal distribution
%\end{remark}
%\end{frame}
%}

\subsection{Connection and curvature}

{
\setbeamertemplate{footline}{}

\begin{frame}
\begin{definition}[Principal bundle connection, {[S.-R.\ F.]}]
\begin{itemize}
	\item On $\mathcal{G}$: Multiplicative Yang-Mills connection
	\item On $\mathcal{P}$: Ehresmann connection
\end{itemize}
\end{definition}

\pause

\begin{remark}[{[S.-R.\ F.]}]
This gives rise to a generalised gauge theory by contracting the involved curvatures.
\end{remark}
\end{frame}

\subsection{Associated bundles and connections}

\begin{frame}
\begin{definition}[Associated bundles, {[C.\ L.-G., S.-R.\ F.]}]
\begin{center}
	\begin{tikzcd}[ampersand replacement=\&]
	\mathcal{P} \arrow[swap]{d}{\pi_{\mathcal{P}}} \arrow{dr}{\phi} 
	\& \arrow[bend right]{l} \mathcal{G} \arrow{d}{\pi_{\mathcal{G}}} \arrow[bend left]{dr}
    \\
	L' \arrow{r}{\Psi} \& L \& \arrow{l}{\pi_{\mathcal{T}}} \mathcal{T}
	\end{tikzcd}
\end{center}
Equivalence relation on $\mathcal{P} \leftindex^{}_{\phi} \times_{\pi_{\mathcal{T}}} \mathcal{T}$
\bas
(p, t) &\sim \mleft( p \cdot g, g^{-1} \cdot t \mright)
\eas
defines the \textbf{associated bundle $\mathcal{P} \tilde{\times} \mathcal{T}$} over $L'$.
\end{definition}
\end{frame}

\begin{frame}
\begin{theorem}[Associated connection, {[C.\ L.-G., S.-R.\ F.]}]
Given a multiplicative Ehresmann connection on $\mathcal{G}$, and related Ehresmann connection on $\mathcal{P}$ and $\mathcal{T}$, then
\bas
\mathup{PT}^{\mathcal{P} \tilde{\times} \mathcal{T}}_\gamma [p, t]
&\coloneqq
\mleft[ \mathup{PT}^{\mathcal{P}}_\gamma(p), \mathup{PT}^{\mathcal{T}}_{\Psi \circ \gamma} (t) \mright]
\eas
is a well-defined connection. 
\end{theorem}

%\begin{remark}
%In the case of Yang-Mills: It is reasonable to assume that the connection on $\mathcal{G}$ is then also Yang-Mills.
%\end{remark}
\end{frame}
}



\section{Foliations as associated bundles}

{
\setbeamertemplate{footline}{}

\begin{frame}
\begin{BackToTheRoots}
Fix a point $l \in L$ with transverse model $\mleft(\mathbb{R}^d, \tau_l\mright)$:
\begin{enumerate}
	\item $G$ a subgroup of $\mathrm{Inn}(\tau_l)$
	\pause
	\item $P$ a principal $G$-bundle, equipped with an ordinary connection
	\pause
	\item $\mathcal{P} \coloneqq (P \times P) \Big/ G$, the \textbf{Atiyah groupoid}
	\pause
	\item $\mathcal{G} \coloneqq (P \times G) \Big/ G$, the \textbf{inner group bundle}
	\pause
	\item $\mathcal{T} \coloneqq \mleft(P \times \mathbb{R}^d\mright) \Big/ G$, the \textbf{normal bundle}
\end{enumerate}
\end{BackToTheRoots}

\pause

\begin{remark}
Think of the induced connection on $\mathcal{T}$ as the $\mathcal{F}$-connection.
\end{remark}

\end{frame}

\begin{frame}
$\mathcal{P}$ comes with two natural projections to $L$, denoted by $t$ and $s$
\begin{center}
	\begin{tikzcd}[ampersand replacement=\&]
	\mathcal{P} \arrow{dr}{s} \arrow[swap]{d}{(t, s)} 
	\& \arrow[bend right]{l} \mathcal{G} \arrow{d} \arrow[bend left]{dr}
    \\
	L \times L \arrow[swap]{r}{\mathrm{pr}_2} \& L \& \arrow{l}{\pi_{\mathcal{T}}} \mathcal{T}
	\end{tikzcd}
\end{center}
\pause
This extends to the associated bundle
\begin{center}
	\begin{tikzcd}[ampersand replacement=\&]
	\mathcal{P} \tilde{\times} \mathcal{T} \arrow[swap]{d}{(t, s)} \arrow{dr}{s} \arrow{r}
	\& \arrow{d}{\pi_{\mathcal{T}}} \mathcal{T}
  \\
	L \times L \arrow[swap]{r}{\mathrm{pr}_2} \& L 
	\end{tikzcd}
\end{center}

\pause

\begin{remark}[{[C.\ L.-G., S.-R.\ F.]}]
Associated connection independent of the choice of $\mathcal{F}$-connection!
\end{remark}
\end{frame}

\begin{frame}
%\begin{definition}[Bisections]
%\begin{center}
	%\begin{tikzcd}[ampersand replacement=\&]
    %\mathcal{P} \leftindex^{}_{s}\times_{\pi_{\mathcal{T}}} \mathcal{T} \arrow{r}
    %\arrow[shift left=1ex]{d}{s} \arrow[shift right=1ex, swap]{d}{t}
    %\& \mathcal{P} \tilde{\times} \mathcal{T}
    %\arrow[shift left=1ex]{d}{s} \arrow[shift right=1ex, swap]{d}{t}
    %\\
    %L \arrow[equal]{r} \& L
   %\end{tikzcd}
%\end{center}
%We call a \textbf{bisection} a section $\sigma$ of $t$ so that $s \circ \sigma$ is a diffeomorphism of $L$. This carries a structure as a group.
%\end{definition}
Explicitly, one possible way:

\begin{remark}
Corresponding to $\mathcal{P}$ there is an Atiyah sequence:
\begin{center}
\begin{tikzcd}[ampersand replacement=\&]
	\mathup{Ad}(P)
	\arrow[hook]{r}
	\&
	\mathup{At}(P)
	\arrow[two heads]{r}
	\&
	\mathup{T}L
	\end{tikzcd}
\end{center}
Via pullback to $\mathcal{T}$ we have a transitive algebroid over $\mathcal{T}$:
\begin{center}
\begin{tikzcd}[ampersand replacement=\&]
	\pi_{\mathcal{T}}^!\mathup{Ad}(P)
	\arrow[hook]{r}
	\&
	\pi_{\mathcal{T}}^!\mathup{At}(P)
	\arrow[two heads]{r}
	\&
	\mathup{T}\mathcal{T}
	\end{tikzcd}
\end{center}
\end{remark}

\pause

\begin{remark}
Observe
\bas
\pi_{\mathcal{T}}^!\mathup{At}(P) \subset 
\mathrm{T} \mleft( \mathcal{P} \leftindex^{}_{s}\times_{\pi_{\mathcal{T}}} \mathcal{T} \mright)
\eas
\end{remark}
\end{frame}

\begin{frame}
$\mathup{Ad}(P)$ and $\mathup{At}(P)$ the adjoint and Atiyah bundle of $P$, respectively:
%$\mathbb{H}$ the horizontal distribution on $\mathcal{P} \tilde{\times} \mathcal{T}$:
\begin{center}
	\begin{tikzcd}[ampersand replacement=\&]
	\Gamma_{\mathup{parallel}}^{\mathup{symmetric}}\mleft( \pi_{\mathcal{T}}^!\mathup{Ad}(P) \mright)
	\arrow[hook]{r} \arrow{d}
	\&
	\Gamma_{\mathup{parallel}}^{\mathup{symmetric}}\mleft( \pi_{\mathcal{T}}^!\mathup{At}(P) \mright)
	\arrow[two heads]{r} \arrow{d}
	\&
	\mathfrak{X}(L) \arrow[equal]{d}
	\\
	\tau
	\arrow[hook]{r}
	\&
	\mathcal{F}_{\mathup{projectable}}
	\arrow[two heads]{r}
	\&
	\mathfrak{X}(L)
	\end{tikzcd}
\end{center}

\begin{remark}
Why Yang-Mills connections?
\bas
\text{Involutive} &\leftrightarrow \text{Connection on } \mathcal{T} \text{ and } \mathcal{G} \text{ Yang-Mills}
\eas
\end{remark}
\end{frame}

}

\section{Conclusion}
{
\setbeamertemplate{footline}{}
\begin{frame}
%\begin{table}[h!]
		%\begin{tabularx}{\textwidth}{X| c c} 
			%\rowcolor{gray}
			%Gauge theory & Structure \\ \hline
			%Yang-Mills & Lie group $G$ \\
			%\rowcolor{Gray}
			%Curved Yang-Mills & Lie group bundle $\mathcal{G}$ \\ 
			%Curved Yang-Mills-Higgs & Lie groupoid $\mathfrak{G}$?
		%\end{tabularx}
%\end{table}

%\begin{remark}
%$\mathup{D}t$ is related to the Atiyah algebroid of $P$, $\mathup{T}P \Big/ G$.
%\end{remark}

\begin{center}
	\begin{tikzcd}[ampersand replacement=\&]
		\& \arrow{dl} \mathcal{F} \arrow{dr} \&
		\\
		\mathup{At}(P) \& \& \text{Associated connection}
	\end{tikzcd}
\end{center}

\begin{remark}[Why \textbf{curved} gauge theory?]
\begin{itemize}
	\item Associated connection invariant under choice of $\mathcal{F}$-connection $\nabla^{\mathcal{F}}$
	\item Associated connection has the form
	\bas
	\nabla^{\mathcal{F}} + A \cdot 
	\eas
	where $A$ is the connection 1-form on $\mathcal{P}$
\end{itemize}
\end{remark}

\end{frame}
}
\thispagestyle{empty}
\topmargin -3.46 cm
\vspace*{\fill}
\begin{center}
\huge \textbf{Thank you!}
\end{center}
\vspace*{\fill}

\end{document}

