\documentclass[12pt, a4paper]{amsart}


\usepackage{amsmath,amssymb,amscd,amsfonts, dsfont}
%%--------------------------------------------------------- Packages Loading --
\usepackage{amssymb, amsmath, amsthm}
\usepackage[top=1truein,bottom=0.5truein,hmargin={0.8truein,0.8truein}]{geometry}
\usepackage{graphicx}
\usepackage{color}
%%-------------------------------------------------------------------- color --
\newcommand{\red}{\color[rgb]{0.8,0,0}}
%\newcommand{\orange}{\color[rgb]{1,0.65,0}}
%\newcommand{\yellow}{\color[rgb]{1,1,0}}
\newcommand{\green}{\color[rgb]{0.1,0.6,0.1}}
\newcommand{\blue}{\color[rgb]{0.1,0.1,0.7}}
%\newcommand{\purple}{\color[rgb]{1,0,1}}
\newtheorem{deff}{Definition}
\newtheorem{pb}[deff]{Problem}
%%%%%%%%%%%%%%%%%%%%%%%%%
%%%%%%%%%%%%%%%%%%%%%%%%%
\begin{document}



\noindent{\Large \bf \green NCTS 2023 Postdoc Symposium}\\       %workshop title
\\
%1
\noindent
{\large \bf \blue Classification of neighbourhoods of leaves of singular foliations \newline \normalsize joint work with Camille Laurent-Gengoux
}

\noindent
{\large \it \red Simon-Raphael Fischer (National Center for Theoretical Sciences)}

\noindent
{\bf Abstract:}\\
{
This talk is about my recent work with Camille Laurent-Gengoux. I will present our results about classifying singular foliations admitting a given leaf $L$ in a manifold $M$ and a given transverse model $(\mathbb{R}^d, \tau)$, where $\mathbb{R}^d$ is the fibre of a normal bundle of $L$ in $M$, and $\tau$ is a singular foliation in $\mathbb{R}^d$ admitting 0 as a leaf. Such a classification is motived by the fact that every foliation $\mathcal{F}$ induces a singular foliation in the fibres of a normal bundle, the \emph{transverse (singular) foliation}, and these transverse foliations at each point in $L$ are canonically isomorphic. These isomorphisms are given by the parallel transport of what one calls $\mathcal{F}$-connections.

The idea of this talk is to recover $\mathcal{F}$ given $(\mathbb{R}^d, \tau)$, and we will see that in a local neighbourhood around $L$ every foliation admitting $(\mathbb{R}^d, \tau)$ as transverse model is given by an associated connection of a curved Yang-Mills gauge theory, a generalised gauge theory I have developed last year. Usually, the horizontal distribution of a flat connection gives rise to a regular foliation, while our condition roughly says that the curvature is related to the field strength of a curved gauge theory. This is a natural enhancement, allowing singular foliations as a consequence, and this construction is naturally invariant of the choice of $\mathcal{F}$-connections.
}\\

\end{document}